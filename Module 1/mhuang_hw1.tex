\documentclass{article}
\linespread{1.3}
\usepackage[margin=50pt]{geometry}
\usepackage{amsmath, amsthm, amssymb, amsthm, tikz, fancyhdr, graphicx, systeme}
\pagestyle{fancy}
\renewcommand{\headrulewidth}{0pt}
\newcommand{\changefont}{\fontsize{15}{15}\selectfont}

\fancypagestyle{firstpageheader}{
  \fancyhead[R]{
    \changefont
    \parbox[t]{4cm}{ % Adjust width as needed
      Michael Huang\\
      EN.625.633.81\\
      Homework 1
    }
  }
}

\begin{document}

\thispagestyle{firstpageheader}
{\Large 

\section*{1.}

\subsection*{(a)} 



\subsection*{(b)}

\begin{align*}
  P(A \cup B \cup C) && \text{Given} \\
  P((A \cup B) \cup C) && \text{Assoc. prop. of union} \\
  P(A \cup B) + P(C) - P((A \cup B) \cap C) && \text{Def. of union} \\
  P(A) + P(B) - P(A \cap B) + P(C) - P((A \cup B) \cap C) && \text{Def. of union} \\
  P(A) + P(B) - P(A \cap B) + P(C) - P((A \cap C) \cup (B \cap C)) && \text{Distr. prop. of intersection} \\
  P(A) + P(B) - P(A \cap B) + P(C) \\ 
  - P(A \cap C) - P(B \cap C) + P(A \cap B \cap A \cap C) && \text{Assoc. prop. of union} \\
  P(A) + P(B) - P(A \cap B) + P(C) \\ 
  - P(A \cap C) - P(B \cap C) + P(A \cap B \cap C) && \text{Prop. of intersection} \\
\end{align*}
which ultimately simplifies to 
\[
  \boxed{P(A) + P(B) + P(C) - P(A \cap B) - P(A \cap C) - P(B \cap C) + P(A \cap B \cap C)}
\]
as we exactly sought to prove.

}

{\Large 

\section*{2.}

We are given \(p = 0.05\). Given the independent trials and the fact that we want to find the minimum number of throws \(n\) such that the chance of having at least one hit in those \(n\) throws is greater than 0.5. As calculating the probability of at least one hit being greater than 0.5 is messy as we need to compute for all terms from 1 to \(n\), let's instead take the complement of this, which is easier to calculate. We set this up as   
\[
  P(\text{one hit in n}) = 1 - P(\text{no hits in n}) = 1 - 0.95^n
\]
We can then solve accordingly:
\begin{align*}
  1 - 0.95^n \geq 0.5 && \text{Given} \\
  - 0.95^n \geq - 0.5 && \text{Arithmetic} \\
  0.95^n \leq 0.5 && \text{Arithmetic} \\
  n \cdot \ln(0.95) \leq \ln(0.5) && \text{Property of logarithm} \\
  n \geq \frac{\ln(0.5)}{\ln(0.95)} && \text{As ln(0.95) is negative} \\
  n \geq 13.513407334 && \text{Arithmetic} \\
\end{align*}
As we need an integer number of throws, we find the minimum number of throws to be \fbox{\textbf{14}}.

}

{\Large 

\section*{3.}

As the outcomes of each player shooting is deemed to be independent, we can model the PMF for the result for the number of attempts by breaking it into two scenarios: Player A makes the first shot, or Player B makes the first shot. Since Player A goes first, we can say that the make is on an odd number of attempts, and Player B making it would be on an even number of attempts. We therefore break this down into
\[
  P(N) = P(N = 2k-1) + P(N = 2k)
\]
where \(N\) is the number of attempts. Given that they make with probability \(p_1\) and \(p_2\), we can break each scenario down using the complements of these probabilities:
\[
  P(N = 2k-1) = ((1 - p_1) \cdot (1 - p_2))^{k-1} \cdot p_1
\]
and 
\[ 
  P(N = 2k) = ((1 - p_1) \cdot (1 - p_2))^{k-1}  \cdot (1 - p_1) \cdot p_2
\]
where \(k \geq 1\). So putting this together, we have the PMF for the make being on the \(N\)th shot being 
\[
  \boxed{
    p_N(k) = ((1 - p_1) \cdot (1 - p_2))^{k-1} \cdot p_1 + ((1 - p_1) \cdot (1 - p_2))^{k-1}  \cdot (1 - p_1) \cdot p_2
  }
\]
for \(k = 1, 2, 3, \) \ldots

}

{\Large 

\section*{4.}

We can use the uniform distribution here, i.e. we can model the point at which we cut with \(X \sim \text{Unif}(0, 1)\). We want to find \(P(\text{Longer Piece} > 2 \cdot \text{Shorter Piece})\),which we can divide into the following scenarios: \\ \\
Scenario 1: We cut above the halfway point, so \(X > \frac{1}{2}\):
\[
  X > 2(1 - X)
\] 
\[
  X > 2 - 2X
\] 
\[
  3X > 2 
\] 
\[
  X > \frac{2}{3}
\] 
\\
Scenario 2: We cut below the halfway point, so \(X < \frac{1}{2}\):
\[
  1 - X > 2X
\]
\[
  1 - X > 2X
\]
\[
  1 > 3X
\]
\[
  X < \frac{1}{3}
\]
Putting this together, we can find the total probability of the scenario to be \\
\[
  P(X \geq \frac{2}{3}) + P(X \leq \frac{1}{3})
\]
which we very intuitively interpret using the Uniform distribution to be 
\[
  \frac{1}{3} + \frac{1}{3} = \boxed{\bf{\frac{2}{3}}}
\]

}

{\Large 

\section*{5.}

We are given the binomial r.v. \(np\), i.e. \(X \sim \text{Binom}(n, p)\). We can first define that we have \(n\) trials in the binomial form and take the expected value by setting up a Bernoulli r.v. with equivalent success rate \(p\). We can then sum the total number of successes into \(X\) by breaking it down into \(X_i\) for each \(i\)th trial, i.e. 
\[
  X = X_1 + X_2 + X_3 + \ldots + X_n
\]
with success being 1 and failure being 0. By definition, we have a bunch of Bernoulli r.v.s that we have now strung together into \(n\) trials, which is exactly what a binomial r.v. is. We take the expected value of \(X\):
\[
  E[X] = E[X_1 + X_2 + X_3 + \ldots + X_n]
\]
And use the property of expectation of r.v.s being the expectation of its components, i.e. linearity of expectation: 
\[
  E[X] = E[X_1] + E[X_2] + E[X_3] + \ldots + E[X_n]
\]
And using the definition of the expected value of a Bernoulli variable, we can find that 
\[
  E[X] = p + p + p + \ldots + p = np
\]
as we sought to find. \\ \\
For variance, we know that each \(X_i\) is independent by definition of the binomial variable, and that \(Var(X_i) = E[X_i^2] - (E[X_i])^2\). We know that for each trial, \(E[X_i^2] = E[X_i]\) since the value is just 0 or 1. Therefore, we know that \(Var(X_i) = E[X_i^2] - (E[X_i])^2 = E[X_i] - (E[X_i])^2 = p - p^2\). Putting this together:
\[
  Var[X] = Var[X_1 + X_2 + X_3 + \ldots + X_n]
\]
\[
  Var(X) = Var(X_1) + Var(X_2) + Var(X_3) + \ldots + Var(X_n)
\]
\[
  Var(X) = p - p^2 + p - p^2 + p - p^2 + \ldots + p - p^2
\]
\[
  Var(X) = n(p - p^2) = np(1-p)
\]
exactly as we sought to find as well.

}

{\Large 

\section*{6.}

We are given an experiment with \(X_1, X_2, \ldots, X_n \sim \text{Unif}[0, 1]\) with all \(X_i\) being independent. To find the pdf of \(X_{(1)} = \text{min}(X_1, \ldots, X_i)\) \\
We can approach this by finding the cdf and taking the derivative of the cdf. By definition, we can find the cdf of the uniform distribution to be 
\[
  F_X(x) = P(X \leq x) = P(X_{(1)} \leq x)
\]
To make this easier, we instead take the complement and continue deriving:
\begin{align*}
  P(X_{(1)} \leq x) = 1 - P(X_{(1)} > x)  && \text{Complement} \\
  = 1 - P(X_1 > x \cap X_2 > x \cap \ldots \cap X_n > x) && \text{Definition of } X_{(1)} \\
  = 1 - P(X_1 > x) \cap P(X_2 > x) \cap \ldots \cap P(X_n > x) && \text{Property of set intersection} \\
  = 1 - (1-x) \cdot (1-x) \cdot \ldots \cdot (1-x) && \text{Definition of uniform distr.} \\
  = 1 - (1-x)^n && \text{Property of set intersection} \\
\end{align*}
Now, to find the pdf, we differentiate:
\begin{align*}
  f_{X_{(1)}}(x) = \frac{d}{dx} (1 - (1-x)^n) && \text{Definition of pdf} \\
  = \frac{d}{dx} (1) - \frac{d}{dx}(1-x)^n && \text{Differentiation} \\
  = 0 - \frac{d}{dx}(1-x)^n && \text{Differentiation} \\
  = - nu^{n-1} \cdot \frac{du}{dx} = - n(1-x)^{n-1} \cdot (0 - 1) && \text{Chain rule, with } u = 1 - x \\
  = \boxed{n(1-x)^{n-1}} && \text{Arithmetic} \\
\end{align*}
for \(0 \leq x \leq 1\), and 0 otherwise.

}

{\Large 

\section*{7.}

\subsection*{(a)} 

We are given \(Y \sim e^X\), with \(X \sim N(0, 1)\). We'll use the methodology from the transformation section. From this, we know that we can determine that the pdf 
\[
  f_Y(y) = f_X(g^{-1}(y))|\frac{d g^{-1}(y)}{dy}|
\]
We know that \(Y \sim g(X) \sim e^X\), so 
\[
  g(y) = e^y
\]
\[
  g^{-1}(y) = \ln(y)
\]
\[
  \frac{d}{dy}g^{-1}(y) = \frac{d}{dy}\ln(y) = \frac{1}{y}
\]
Now we can substitute accordingly. We know as well by definition that for \(X \sim N(0, 1)\), \(f_X(x) = \frac{1}{\sqrt{2\pi}}e^{-\frac{x^2}{2}}\). Putting this all together:
\[
  f_Y(y) = f_X(g^{-1}(y))|\frac{d g^{-1}(y)}{dy}|
\]
\[
  f_Y(y) = f_X(\ln(y))|\frac{1}{y}|
\]
\[
  f_Y(y) = \frac{1}{\sqrt{2\pi}}e^{-\frac{\ln(y)^2}{2}}|\frac{1}{y}|
\]
\[
  \boxed{f_Y(y) = \frac{1}{y} \cdot \frac{1}{\sqrt{2\pi}}e^{-\frac{\ln(y)^2}{2}}}
\]
for \(y > 0\), as by definition \(Y \sim e^X > 0\) always, and 0 otherwise.

\subsection*{(b)}

We want to find \(E[Y] = E[e^X]\). 
% By definition, since for \(X \sim N(\mu, \sigma^2)\), \(E[X] = \mu + \sigma^2\); in this case, \(X \sim N(0,1)\) so \(E[X] = 0 + 1^2 = 1\)
We know that in general, \(E[Y] = \int_{-\infty}^{\infty} y f_Y(y) dy\). Given the pdf we found in part (a), let's start by plugging in:
\[
  E[Y] = \int_{-\infty}^{\infty} y \cdot \frac{1}{y} \cdot \frac{1}{\sqrt{2\pi}}e^{-\frac{\ln(y)^2}{2}} dy
\]
\[
  E[Y] = \frac{1}{\sqrt{2\pi}}\int_{-\infty}^{\infty} e^{-\frac{\ln(y)^2}{2}} dy
\]
\[
  E[Y] = \frac{1}{\sqrt{2\pi}}\int_{-\infty}^{\infty} e^{-\frac{\ln(y)^2}{2}} dy
\]
We can perform integration by substitution. Using \(x = \ln(y)\) (and subsequently, \(y = e^x\) and \(dy = e^xdx\)):
\[
  E[Y] = \frac{1}{\sqrt{2\pi}}\int_{-\infty}^{\infty} e^{-\frac{x^2}{2}} \cdot e^x dx
\]
\[
  E[Y] = \frac{1}{\sqrt{2\pi}}\int_{-\infty}^{\infty} e^{x-\frac{x^2}{2}} dx
\]
\[
  E[Y] = \frac{1}{\sqrt{2\pi}}\int_{-\infty}^{\infty} e^{x-\frac{x^2}{2}} dx
\]
We want to make the exponent of the \(e\) term more interpretible, in order to get something with the form of the normal pdf, as we know that generally for \(Z \sim N(\mu, \sigma^2)\), \(f_Z(z) = \frac{1}{\sqrt{2\pi}} e^{-\frac{(z - \mu)^2}{2\sigma^2}} \). We complete the square accordingly to get to this form:
\[
  x - \frac{x^2}{2} = -\frac{1}{2}(x^2 - 2x)
\]
Set up with \((x-1)^2 = x^2 + 1 - 2x\):
\[
  x^2 - 2x + 1 = (x-1)^2
\]
\[
  x^2 - 2x = (x-1)^2 - 1
\]
\[
  -\frac{1}{2}(x^2 - 2x) = -\frac{1}{2}((x-1)^2 - 1) = \frac{1}{2} - \frac{(x-1)^2}{2}
\]
So substituting back:
\[
  E[Y] = \frac{1}{\sqrt{2\pi}}\int_{-\infty}^{\infty} e^{\frac{1}{2} - \frac{(x-1)^2}{2}} dx
\]
\[
  E[Y] = e^{\frac{1}{2}} \cdot \frac{1}{\sqrt{2\pi}}\int_{-\infty}^{\infty} e^{-\frac{(x-1)^2}{2}} dx
\]
And we use the fact that we know the form of the normal pdf to find it is a pdf with \(\mu=1\), \(\sigma=1\). The integral of this kind of distribution is simply 1, so:
\[
  E[Y] = e^{\frac{1}{2}} \cdot 1 = \boxed{e^{\frac{1}{2}}}
\]

}

\end{document}